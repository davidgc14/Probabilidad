\documentclass[fleqn]{article}

%\pgfplotsset{compat=1.17}

\usepackage{mathexam}
\usepackage{amsmath}
\usepackage{amsfonts}
\usepackage{graphicx}
\usepackage{systeme}
\usepackage{microtype}
\usepackage{multirow}
\usepackage{pgfplots}
\usepackage{listings}
\usepackage{tikz}
\usepackage{dsfont} %Numeros reales, naturales...

%\graphicspath{{images/}}
\newcommand*{\QED}{\hfill\ensuremath{\square}}

%Estructura de ecuaciones
\setlength{\textwidth}{15cm} \setlength{\oddsidemargin}{5mm}
\setlength{\textheight}{23cm} \setlength{\topmargin}{-1cm}



\author{David García Curbelo}
\title{Probabilidad}
\date{Problema 6 Tema 4 propuesto}

\pagestyle{empty}



\def\R{\mathds{R}}
\def\sup{$^2$}

\begin{document}
    \maketitle
    \setcounter{page}{1}
    \pagestyle{plain}

    \textbf{Ejercicio 6. } \textit{Dadas las siguientes distribuciones de probabilidad bidimensionales discretas}\\

    \begin{tabular}{|c|c c c|}
        \hline 
        $X|Y$ & 0 & 1 & 2  \\ \hline
        0 & 1/5 & 0 & 0\\
        2 & 0 & 1/5 & 0\\
        3 & 1/5 & 0 & 1/5\\
        4 & 0 & 0 & 1/5\\
        \hline
    \end{tabular}
    \begin{tabular}{|c|c c c|}
        \hline 
        $X'|Y$ & 0 & 1 & 2  \\ \hline
        0 & 1/5 & 0 & 1/5\\
        2 & 0 & 1/5 & 0\\
        3 & 1/5 & 0 & 0\\
        4 & 0 & 0 & 1/5\\
        \hline
    \end{tabular}

    \textit{Calcular:}

    \textit{a) Curvas de regresión y errores cuadráticos medios asociados}\\
    \fbox{$Y/X$}\\
    \begin{equation*}
        \begin{aligned}
            &E[Y/X=0] = 0\cdot 1 + 1\cdot 0 + 2\cdot 0 = 0\\
            &E[Y/X=2] = 0+ 1 + 0 = 1\\
            &E[Y/X=3] = 0 + 0 + 2\cdot\frac{1}{2} = 1\\
            &E[Y/X=4] = 0 + 0 + 2\cdot 1 = 2\\
        \end{aligned}\\
        E[Y/X]= \left\{
        \begin{aligned}
            &0 \xrightarrow{g} 0 = g(1), \quad P_{g(1)} = 1/5 \\
            &2 \xrightarrow{g} 1 = g(2), \quad P_{g(2)} = 1/5 \\
            &3 \xrightarrow{g} 1 = g(3), \quad P_{g(3)} = 2/5 \\
            &4 \xrightarrow{g} 2 = g(4), \quad P_{g(4)} = 1/5 \\
        \end{aligned}
        \right.
    \end{equation*}
    Con el objetivo de calcular el error cuadrático medio, obtendremos primero la varianza, que viene dada por 
    $Var(Y/X)=E[Y^2/X] - \left(E[Y/X]\right)^2$. Para ello necesitamos unos cálculos previos

    \begin{equation*}
        \begin{aligned}
            &E[Y^2/X=0] = 0\\
            &E[Y^2/X=2] =1 \dot 1 = 1\\
            &E[Y^2/X=3] = 4\cdot\frac{1}{2} = 2\\
            &E[Y^2/X=4] = 4\cdot 1 = 4\\
        \end{aligned}\\
        E[Y/X]= \left\{
        \begin{aligned}
            &0 \xrightarrow{h} 0 = h(1), \quad P_{h(1)} = 1/5 \\
            &2 \xrightarrow{h} 1 = h(2), \quad P_{h(2)} = 1/5 \\
            &3 \xrightarrow{h} 2 = h(3), \quad P_{h(3)} = 1/5 \\
            &4 \xrightarrow{h} 4 = h(4), \quad P_{h(4)} = 1/5 \\
        \end{aligned}
        \right.
    \end{equation*}

    \begin{equation*}
        \begin{aligned}
            &Var[Y/X=0] = 0-0^2 = 0, \quad P_{f(1)} = 1/5\\
            &Var[Y/X=2] = 1-1^2 = 0, \quad P_{f(2)} = 1/5\\
            &Var[Y/X=3] = 2-1^2 = 1, \quad P_{f(3)} = 2/5\\
            &Var[Y/X=4] = 4-2^2 = 0, \quad P_{f(4)} = 1/5\\
        \end{aligned}
    \end{equation*}

    Entonces tenemos ya el error cuadrático medio
    $$ECM(Y/X) = E[Var(Y/X)] = \frac{2}{5}$$\\


    \fbox{$X/Y$}\\
    \begin{equation*}
        \begin{aligned}
            &E[X/Y=0] = 0 + 3 \cdot 1/2 = 3/2\\
            &E[X/Y=1] = 2 \cdot 1  = 2\\
            &E[X/Y=2] = 3\cdot 1/2 + 4 \cdot 1/2 = 7/2\\
        \end{aligned}\\
        E[X/Y]= \left\{
        \begin{aligned}
            &0 \xrightarrow{g} 3/2 = g(0), \quad P_{g(0)} = 2/5 \\
            &1 \xrightarrow{g} 2 = g(1), \quad P_{g(1)} = 1/5 \\
            &2 \xrightarrow{g} 7/2 = g(2), \quad P_{g(2)} = 2/5 \\
        \end{aligned}
        \right.
    \end{equation*}
    Con el objetivo de calcular el error cuadrático medio, obtendremos primero la varianza, que viene dada por 
    $Var(X/Y)=E[X^2/Y] - \left(E[X/Y]\right)^2$. Para ello necesitamos unos cálculos previos

    \begin{equation*}
        \begin{aligned}
            &E[X^2/Y=0] = 9/2\\
            &E[X^2/Y=1] = 4\\
            &E[X^2/Y=2] = 25/2\\
        \end{aligned}\\
        \begin{aligned}
            &Var[X/Y=0] = 9/4\\
            &Var[X/Y=1] = 0\\
            &Var[X/Y=2] = 1/4\\
        \end{aligned}
    \end{equation*}

    Entonces tenemos ya el error cuadrático medio
    $$ECM(X/Y) = E[Var(X/Y)] = 1$$\\

    \textit{b) Razones de correlación}

    $$\eta _{X/Y}^2 = \frac{Var(E[X/Y])}{Var(X)} = 0.456521739$$
    $$\eta _{Y/X}^2 = \frac{Var(E[Y/X])}{Var(Y)} = 0.5$$

    \begin{equation*}
        E[X] = 12/5
        E[X^2] = 38/5
        E[Y] = 1
        E[Y^2] = 9/5
    \end{equation*}
    \begin{equation*}
        Var[X] = E[X^2]-(E[X])^2 = 46/25
        Var[Y] = E[Y^2]-(E[Y])^2 = 4/5
    \end{equation*}\\

    \textit{c) Rectas de regresión $Y/X$ y $X/Y$}
    $$\hat{y}= E[Y] + \frac{Cov(X,Y)}{Var(X)}(X-E[X]) = 1 + \frac{4/5}{46/25}(X-\frac{12}{5})$$
    $$\hat{x}= E[X] + \frac{Cov(Y,X)}{Var(Y)}(Y-E[Y]) = \frac{12}{5} + \frac{4/5}{4/5}(Y-1)$$
    Teniendo en cuenta $Cov(X,Y) = E[XY]-(E[X]E[Y])=4/5$, con $E[XY]=16/5$\\

    \textit{d) Coeficiente de correlación lineal}
    $$P_{X,Y} = \frac{Cov(X,Y)}{\sqrt{Var(X)Var(Y)}} = \frac{4/5}{\sqrt{46/25 \cdot 4/5}} = 0.659380473$$\\

    \textit{f) Razones de correlación}

    $$\eta_{X/Y}^2 = 1 -\frac{5/2}{64/25} = 0.0234375$$
    \begin{equation*}
        Var[X'] = E[X'^2]-(E[X'])^2 = 64/25\\ \\
        E[X']= 9/5\\ \\
        E[X'^2] = 29/5\\ \\
        Var[Y]= 4/5\\ \\
    \end{equation*}
    $$\eta_{Y/X}^2 = 1 -\frac{2/5}{4/5} = 1/2$$\\

    \textit{g) Rectas de regresión}
    $$\hat{y}= E[Y] + \frac{Cov(X',Y)}{Var(X')}(X'-E[X']) = \frac{7}{10}X'-\frac{13}{50}$$
    $$\hat{x'}= E[X'] + \frac{Cov(Y,X')}{Var(Y)}(Y-E[Y]) = \frac{506}{320} + \frac{14}{64}Y$$
    Teniendo en cuenta $Cov(X',Y) = E[X'Y]-(E[X']E[Y])=1/5$, con $E[X'Y]=2$\\

    \textit{h) Coeficiente de correlación lineal}
    $$P_{X',Y} = \frac{Cov(X',Y)}{\sqrt{Var(X')Var(Y)}} = \frac{1/5}{\sqrt{64/25 \cdot 4/5}} = 0.139754248$$\\

    \textit{i) ¿Cuál se aproxima más a la variable aleatoria $Y$?}

    X aproxima mejor a la variable aleatoria Y, ya que el coeficiente de correlación lineal es un valor más cercano a 1.


\end{document}