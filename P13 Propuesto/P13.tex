\documentclass[a4paper, 12pt]{article}

\usepackage{mathexam}
\usepackage{amsmath}
\usepackage{amsfonts}
\usepackage{graphicx}
\usepackage{systeme}
\usepackage{microtype}
\usepackage{multirow}
\usepackage{pgfplots}
\usepackage{listings}
\usepackage{tikz}
\usepackage{dsfont} %Numeros reales, naturales...
\usepackage{verbatimbox} %comentarios

%\graphicspath{{images/}}
\newcommand*{\qed}{\hfill\ensuremath{\square}}

%Estructura de ecuaciones
\setlength{\textwidth}{15cm} \setlength{\oddsidemargin}{5mm}
\setlength{\textheight}{23cm} \setlength{\topmargin}{-1cm}



\author{David García Curbelo}
\title{Probabilidad}
\date{Problema P12-P13 propuesto}

\pagestyle{empty}



\def\R{\mathds{R}}
\def\sup{$^2$}

\begin{document}
    \maketitle
    \setcounter{page}{1}
    \pagestyle{plain}

    {\bf{Actividad 1: }}{\textit{Sea (X,Y) un vector aleatorio con función de distribución de probabilidad dada por 
    $$f_{(X,Y)}(x,y)=ky^2, \quad -1 \leq x \leq 0; -1\leq x\leq y\leq x^2 \quad \acute{o} \quad 0\leq x\leq 1;0\leq x^2 \leq y\leq 1$$
    Calcular la constante $k$ para que la función anterior sea una función de densidad de probabilidad. Calcular además 
    la función de densidad de probabilidad.}}\\

    Para ver que la función de distribución dada está bien definida ha de cumplir las siguientes condiciones:
    \begin{itemize}
        \item Ha de ser no negativa
        \item $\int _{-\infty}^\infty \int _{-\infty}^\infty f_{(X,Y)}(x,y) dydx = 1$
        \item Ha de ser continua
    \end{itemize}

    Vemos que para que sea no negativa partimos de que $k\geq 0$. Calculemos la integral definida para el primer dominio
    (primera condición nombrada en el enunciado, $-1 \leq x \leq 0; -1\leq x\leq y\leq x^2$)
    $$\int _{-\infty}^\infty \int _{-\infty}^\infty f_{(X,Y)}(x,y) dydx = \int _{-1}^0 \int _{u}^{u^2} f_{(X,Y)}(u,v) dvdu = 
    \int _{-1}^0 \left[\frac{k}{3}v^3\right]_{u}^{u^2} du = $$
    $$ = \frac{k}{3} \int _{-1}^0 u^6 - u^3 du = \frac{k}{3} \left[\frac{u^7}{7} - \frac{u^4}{4}\right]_{-1}^0 =
    \frac{k}{3} \left(\frac{1}{7} + \frac{1}{4}\right) = \frac{11k}{84}$$ 
    
    Calculemos ahora la integral definida para el segundo dominio
    (segunda condición nombrada en el enunciado, $0\leq x\leq 1;0\leq x^2 \leq y\leq 1$)
    $$\int _{-\infty}^\infty \int _{-\infty}^\infty f_{(X,Y)}(x,y) dydx = \int _{0}^1 \int _{u^2}^{1} f_{(X,Y)}(u,v) dvdu = 
    \int _{0}^1 \left[\frac{k}{3}v^3\right]_{u^2}^{1} du = $$
    $$ \frac{k}{3} \int _{0}^1 1-u^6 du = \frac{k}{3} \left[u-\frac{u^7}{7}\right]_{0}^1 = \frac{k}{3} \left[1-\frac{1}{7}\right] =
    \frac{6k}{21}=\frac{2k}{7}$$

    Como sabemos, la suma de ambos dominios ha de ser 1, por ello
    $$\frac{2k}{7}+\frac{11k}{84}=1$$
    y por tanto concluimos que el valor de $k$ viene dado por $k=\frac{12}{5}$.\\ \\ 

    Para el siguiente apartado partimos de la función de distribución $f_{(X,Y)}(x,y)=\frac{12}{5}y^2$ en el dominio 
    antes mencionado. Nuestra función de densidad estará definida en dos partes, referentes a los dos dominios diferentes
    en los que trabajaremos, de los que extraemos los siguientes recintos:\\ \\

    Recinto 1: $R_1=\left\{(x,y) \in \R^2 \quad / \quad 0\leq x^2 \leq y <1\right\}$

    $$F(x,y)=\int _0^x \int _{u^2}^y \frac{12}{5}v^2 dvdu + \int _{-\sqrt{y}}^0\int _{u}^{u^2} \frac{12}{5}v^2 dvdu + \int_{-1}^{-\sqrt{y}} \int _u^y \frac{12}{5}v^2 dvdu= 
    $$$$\frac{12}{15} \left(\int _0^x \left[y^3-u^6\right] du + \int _{-\sqrt{y}}^0 \left[u^6-u^3\right] du + \int _{-1}^{-\sqrt{y}} \left[y^3-u^3\right] du\right) =
    $$$$\frac{4}{35}\left(y^3x-x^7-6y^{7/2} + y^3+ \frac{1}{4}\right), \quad \forall (x,y) \in R_1$$\\ \\

    Recinto 2: $R_2=\left\{(x,y) \in \R^2 \quad / \quad -1\leq x \leq 0, x^2 <y<1\right\}$

    $$F(x,y)=\int _{-1}^{-\sqrt{y}} \int _{u}^y \frac{12}{5}v^2 dvdu + \int _{-\sqrt{y}}^x\int _{u}^{u^2} \frac{12}{5}v^2 dvdu= 
    \frac{12}{15} \left(\int _{-1}^{-\sqrt{y}} \left[y^3-u^3\right] du + \int _{-\sqrt{y}}^x \left[u^6-u^3\right] du\right) =
    $$$$\frac{4}{35}\left(-42y^3\sqrt{y}+y^3 + x^7-\frac{x^4}{4}+\frac{1}{4}\right), \quad \forall (x,y) \in R_2$$\\ \\

    Recnito 3: $R_3=\left\{(x,y) \in \R^2 \quad / \quad -1\leq x \leq 0, x <y<x^2\right\}$

    $$F(x,y)=\int _{-1}^{x} \int _{u}^y \frac{12}{5}v^2 dvdu = 
    \frac{12}{15} \int _{-1}^{-\sqrt{y}} \left[y^3-u^3\right] du =
    \frac{4}{5}\left(y^3x + y^3 -\frac{x^4}{4}+1\right), \quad \forall (x,y) \in R_3$$\\ \\

    Recinto 4: $R_4=\left\{(x,y) \in \R^2 \quad / \quad -1\leq y \leq x, y<0\right\}$

    $$F(x,y)=\int _{-1}^{y} \int _{-1}^y \frac{12}{5}v^2 dvdu = 
    \frac{12}{15} \int _{-1}^{y} \left[y^3+1\right] du =
    \frac{4}{5}\left(y^4+y^3+y+1\right), \quad \forall (x,y) \in R_4$$\\ \\

    Recinto 5: $R_5=\left\{(x,y) \in \R^2 \quad / \quad 0\leq y \leq 1, y<x^2\right\}$

    $$F(x,y)=\int _{-1}^{-\sqrt{y}} \int _{u}^y \frac{12}{5}v^2 dvdu + \int _{-\sqrt{y}}^0\int _{u}^{u^2} \frac{12}{5}v^2 dvdu + \int _{0}^{\sqrt{y}} \int _{u^2}^y \frac{12}{5}v^2 dvdu= 
    $$$$\frac{12}{15} \left(\int _{-1}^{-\sqrt{y}} \left[y^3-u^3\right] du + \int _{-\sqrt{y}}^0 \left[u^6-u^3\right]du + \int _{0}^{\sqrt{y}} \left[y^3-u^6\right] du\right) =
    \frac{1}{5}\left(4y^3+1\right), \quad \forall (x,y) \in R_5$$\\ \\

    Recinto 6: $R_6=\left\{(x,y) \in \R^2 \quad / \quad 0\leq x \leq 1, 1\leq y\right\}$

    $$F(x,y)=\int _{-1}^{0} \int _{u}^{u^2} \frac{12}{5}v^2 dvdu + \int _{0}^x\int _{u^2}^{1} \frac{12}{5}v^2 dvdu= 
    \frac{12}{15} \left(\frac{11}{28} + \int _{-\sqrt{0}}^x \left[1-u^6\right] du\right) =
    $$$$\frac{4}{35}\left(7x-x^7\right) + \frac{11}{35}, \quad \forall (x,y) \in R_6$$\\ \\

    Recinto 7: $R_7=\left\{(x,y) \in \R^2 \quad / \quad -1 \leq x \leq 0, 1\leq y\right\}$

    $$F(x,y)=\int _{-1}^{x} \int _{u}^{u^2} \frac{12}{5}v^2 dvdu = 
    \frac{12}{15} \int _{-1}^{x} \left[u^6-u^3\right] du =
    \frac{1}{35}\left(4x^7-7x^4+11\right), \quad \forall (x,y) \in R_7$$\\ \\

    Recinto 8: $R_8=\left\{(x,y) \in \R^2 \quad / \quad x \leq -1, y\leq -1\right\}$

    $$F(x,y)=0, \quad \forall (x,y) \in R_8$$\\ \\

    Recinto 9: $R_9=\left\{(x,y) \in \R^2 \quad / \quad 1 \leq x, 1\leq y\right\}$

    $$F(x,y)=1, \quad \forall (x,y) \in R_9$$
\begin{comment}

    Para el primer recinto estudiemos la siguiente integral:
    $$F_{(X,Y)}(x,y)=\int _{-\infty}^x \int _{-\infty}^y f_{(X,Y)}(u,v) dvdu = \int _{-1}^x \int _{u}^y \frac{12}{5}v^2 dvdu =
    \frac{12}{5} \int _{-1}^x \left[\frac{v^3}{3}\right]_{u}^y du = $$
    $$\frac{12}{15} \int _{-1}^x y^3-u^3 du = 
    \frac{12}{15} \left[y^3u-\frac{u^4}{4}\right]_{-1}^x = \frac{12}{15} \left[y^3x - \frac{x^4}{4} + y^3 + \frac{1}{4}\right]$$
    
    Para el segundo recinto tenemos:
    $$F_{(X,Y)}(x,y)=\int _{-\infty}^x \int _{-\infty}^y f_{(X,Y)}(u,v) dvdu = \int _{0}^x \int _{u^2}^y \frac{12}{5}v^2 dvdu =
    \frac{12}{5} \int _{0}^x \left[\frac{v^3}{3}\right]_{u^2}^y du = $$
    $$\frac{12}{15} \int _{0}^x y^3-u^6 du = \frac{12}{15} \left[y^3u-\frac{u^7}{7}\right]_{0}^x = \frac{12}{105} \left[7y^3x-x^7\right]$$

    Por lo tanto obtenemos la expresión de la función de densidad de probabilidad
    
    \begin{equation*} F_{(X,Y)}(x,y)=
        \left\{
            \begin{aligned}
                \frac{12}{15} \left[y^3x - \frac{x^4}{4} + y^3 + \frac{1}{4}\right], \quad -1 \leq x \leq 0; -1\leq x\leq y\leq x^2 \\
                \frac{12}{105} \left[7y^3x-x^7\right], \quad 0\leq x\leq 1;0\leq x^2 \leq y\leq 1
            \end{aligned}
        \right.
    \end{equation*}
\end{comment}

    \newpage

    {\bf{Actividad 2: }} {\textit{Sea $(X, Y)$ un vector aleatorio, cuya función de densidad de probabilidad conjunta se calcula 
    como producto de las funciones de densidad de probabilidad de dos exponenciales con parámetros $\lambda$ y $\mu$ 
    respectivamente. Es decir
    $$f_{(X,Y)}(x,y) = \lambda \mu \exp (-(\lambda x + \mu y)), \quad \forall (x,y) \in \R _+^2$$
    Calcular la Función de Distribución de Probabilidad de $(X, Y)$.}}\\

    La función de distribución de nuestro vector aleatorio $(X,Y)$ se define como una función 
    $F_{(X,Y)}:\R^2 \rightarrow[0,1]$ dada por
    $$F_{(X,Y)}(x,y) = P_{(X,Y)}((-\infty, x]\times(-\infty, y]) = P(X\leq x, Y\leq y), \quad (x,y) \in \R _+^2$$
    donde podemos deducir que se trata de usar la igualdad 
    $P(X\leq x, Y\leq y)=\int _{-\infty}^x \int _{-\infty}^y f_{(X,Y)}(u,v) du dv$
    por ser un vector aleatorio continuo. Por tanto podemos considerar
    $$F_{(X,Y)}(x,y) = \int _{-\infty}^x \int _{-\infty}^y f_{(X,Y)}(u,v) du dv$$
    $$\int _{-\infty}^x \int _{-\infty}^y f_{(X,Y)}(u,v) dv du = 
    \int _{-\infty}^x \int _{-\infty}^y \lambda \mu \exp (-(\lambda u + \mu v)) dv du = $$

    $$= \lambda \mu \int _{-\infty}^x \int _{-\infty}^y  e^{-(\lambda u + \mu v)} dv du$$
    Como $(x,y) \in \R _+^2$, podemos redefinir las integrales de la sguiente manera:
    $$\lambda \mu \int _{0}^x \int _{0}^y  e^{-(\lambda u + \mu v)} dv du = 
    \lambda \int _{0}^x \left[-e^{-(\lambda u + \mu v)}\right]_{0}^y du = 
    \lambda \int _{0}^x e^{-\lambda u} \left(e^{-\mu y}-1\right)du =$$
    $$= \left[e^{-\lambda u} \left(e^{-\mu y}-1\right)\right]_{0}^x = \left(e^{-\mu y}-1\right)\left(e^{-\lambda x}-1\right)$$

    Y por lo tanto la función de distribución de probabilidad viene dada por 
    $$F_{(X,Y)}(x,y) =\left(e^{-\mu y}-1\right)\left(e^{-\lambda x}-1\right), \quad (x,y) \in \R _+^2$$
\end{document}