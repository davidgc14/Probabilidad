\documentclass[a4paper, 12pt]{article}

\usepackage{mathexam}
\usepackage{amsmath}
\usepackage{amsfonts}
\usepackage{graphicx}
\usepackage{systeme}
\usepackage{microtype}
\usepackage{multirow}
\usepackage{pgfplots}
\usepackage{listings}
\usepackage{tikz}
\usepackage{dsfont} %Numeros reales, naturales...
\usepackage{verbatimbox} %comentarios

%\graphicspath{{images/}}
\newcommand*{\qed}{\hfill\ensuremath{\square}}

%Estructura de ecuaciones
\setlength{\textwidth}{15cm} \setlength{\oddsidemargin}{5mm}
\setlength{\textheight}{23cm} \setlength{\topmargin}{-1cm}



\author{David García Curbelo}
\title{Probabilidad}
\date{Actividad teórica 26 de abril}

\pagestyle{empty}


\def\R{\mathds{R}}
\def\sup{$^2$}

\begin{document}
    \maketitle
    \setcounter{page}{1}
    \pagestyle{plain}

    {\bf{Demostración 1: }}
    Sea $X$ una variable aleatoria sobre un espacio probabilístico $\left(\Omega,\mathcal{A}, P\right)$.
    Consideramos $Y$ una variable aleatoria cualquiera, distinguimos dos casos:
    \begin{itemize}
        \item {\it{Caso 1: }} $Y$ sea discreta, donde $P_X(x,y)=P[X=x, Y=y]$. Consideremos $x=c$
        $$\Rightarrow P_X(x,y)=P[X=c, Y=y] = P[Y=y] = P[Y=y]\cdot P[X=c] = P_X(x)P_Y(y)$$
        Consideremos ahora $x \neq c$
        $$\Rightarrow P_X(x,y)=P[X=x, Y=y] = P[Y=y] = 0 = P[Y=y]\cdot P[X=x] = P_X(x)P_Y(y)$$
        Por lo tanto obtenemos que $\forall x,y \quad P_X(x,y)=P_X(x)P_Y(y)\Rightarrow X$ e $Y$ independientes.

        \item {\it{Caso 2: }} $Y$ sea continua. Sean $B_1$ y $B_2 \in \mathcal{B}$,
        con $P_X(B_1 \times B_2)=P(X \in B_1,Y \in B_2)$. Si $c\in B_1$
        $$\Rightarrow P(X \in B_1,Y \in B_2) = P(Y \in B_2)=P(X\in B_1)P(Y \in B_2)=P_X(B_1)P_Y(B_2)$$
        $$c \not\in B_1 \Rightarrow P(X \in B_1,Y \in B_2) = 0 = P(X\in B_1)P(Y \in B_2)=P_X(B_1)P_Y(B_2)$$
    \end{itemize}
    Por lo tanto concluimos que $X$ e $Y$ son independientes, cualquiera que sea $Y$

    \newpage

    {\bf{Demostración 2: }}
    Definimos $\underline{X}=(X_1,...,X_n)$, donde $X_1,...,X_n$ son independientes.
    Sea $(X_{i_1},...,X_{i_k})/\left\{i_1,...,i_k\right\} \subset \left\{1,...,n\right\}$.
    Veamos que este subconjunto de vectores aleatorios tambien lo son. Distinguimos 2 casos:

    \begin{itemize}
        \item {\it{Caso 1: }} $\underline{X}$ sea un vector aleatorio discreto.
        $$P_{X_{i_1},...,X_{i_n}}(x_{i_1},...,x_{i_n})=\sum _{x_l \in E_{X_l}; l\neq i_1,...,i_k} 
        P_{X_{1},...,X_{n}}(x_{1},...,x_{n})=
        $$$$\sum _{x_l \in E_{X_l}; l\neq i_1,...,i_k} 
        P_{X_1}(x_1)\cdots P_{X_n}(x_n)= P_{X_{i_1}}(x_{i_1})\cdots P_{X_{i_k}}(x_{i_k})$$
        \item {\it{Caso 2: }} $\underline{X}$ sea un vector aleatorio continuo.
        $$f_{X_{i_1},...,X_{i_n}}(x_{i_1},...,x_{i_n})=\int f_X(x_{i_1},...,x_{i_n}) \prod _{x_l \in E_{X_l}; l\neq i_1,...,i_k}dx_l=
        $$$$\int f_{X_1}(x_1) \cdots f_{X_n}(x_n) \prod _{x_l \in E_{X_l}; l\neq i_1,...,i_k}dx_l=
        f_{X_{i_1}}(x_{i_1})\cdots f_{X_{i_k}}(x_{i_k})$$
    \end{itemize}

    \newpage

    {\bf{Demostración 3: }} 
    Definimos $\underline{X}=(X_1,...,X_n)$, donde $X_1,...,X_n$ son independientes.
    Consideramos $(X_{i_1},...,X_{i_k})$ y $(X_{j_1},...,X_{j_p})/ \{i_1,...,i_k\},\{j_1,...,j_p\} \subset \{1,...,n\}$,
    $i_m \neq j_l \quad m=1,...,k \quad l=1,...,p \quad k+p=n$. Distinguimos dos casos:
    \begin{itemize}
        \item {\it{Caso 1: }}  $\underline{X}$ sea un vector aleatorio discreto.
        $$P_{X_{j_1},...,X_{j_p}}(x_{j_1},..,x_{j_p}/X_{i_1}=Y_{i_1},...,X_{i_k}=Y_{i_k})=
        \frac{P_{X_1,...,X_n}(x_{j_1},..,x_{j_p}/Y_{i_1},...,Y_{i_k})}{P_{X_{i_1},...,X_{i_k}}(Y_{i_1},...,Y_{i_k})}=
        $$$$P_{X_{j_1}}(x_{j_1}) \cdots P_{X_{j_p}}(x_{j_p})=P_{X_{j_1},...,X_{j_p}}(x_{j_1},..,x_{j_p})
        $$$$\Rightarrow \mathnormal{Distribucion marginal de } X_{j_1},...,X_{j_1}$$
        
        \item {\it{Caso 2: }}  $\underline{X}$ sea un vector aleatorio continuo.
        $$f_{X_{j_1},...,X_{j_p}}(x_{j_1},..,x_{j_p}/X_{i_1}=Y_{i_1},...,X_{i_k}=Y_{i_k})=
        \frac{f_{X_1,...,X_n}(x_{j_1},..,x_{j_p}/Y_{i_1},...,Y_{i_k})}{f_{X_{i_1},...,X_{i_k}}(Y_{i_1},...,Y_{i_k})}=
        $$$$f_{X_{j_1}}(x_{j_1}) \cdots f_{X_{j_p}}(x_{j_p})=f_{X_{j_1}} \cdots f_{X_{j_p}}(x_{j_1},..,x_{j_p})
        $$$$\Rightarrow \mathnormal{Distribucion marginal de } X_{j_1},...,X_{j_1}$$
    \end{itemize}

    \newpage

    {\bf{Demostración 4: }} 
    Definimos $\underline{X}=(X_1,...,X_n)$, donde $X_1,...,X_n$ son independientes.
    $\Rightarrow M_{X_1,...,X_n}(t_1,...,t_n)=M_{X_1}(t_1) \cdots M_{X_n}(t_n)$ para el caso discreto.
    $$M_{X_1,...,X_n}(t_1,...,t_n)=E\left[exp\left(\sum_{i=1}^n t_iX_i\right)\right]
    \sum _{(x_1,...,x_n) \in E_{\underline{X}}} exp\left(\sum_{i=1}^n t_iX_i\right)$$

    $$P_X(x_1,...,x_n)=\sum _{x_1 \in E_{X_1}}exp(t_1 x_1)P_{X_1}(x_1) + ... + \sum _{x_n \in E_{X_n}}exp(t_n x_n)P_{X_n}(x_n)=
    $$$$M_{X_1}(t_1) \cdots M_{X_n}(t_n) \quad \forall (t_1,...,t_n) \in (-a_1,b_1)\times ... \times (-a_n,b_n) /
    $$$$a_i,b_i \in \R^+, \quad i=1,...,n$$

\end{document}
