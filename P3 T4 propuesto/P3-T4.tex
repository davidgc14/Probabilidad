\documentclass[fleqn]{article}

%\pgfplotsset{compat=1.17}

\usepackage{mathexam}
\usepackage{amsmath}
\usepackage{amsfonts}
\usepackage{graphicx}
\usepackage{systeme}
\usepackage{microtype}
\usepackage{multirow}
\usepackage{pgfplots}
\usepackage{listings}
\usepackage{tikz}
\usepackage{dsfont} %Numeros reales, naturales...

%\graphicspath{{images/}}
\newcommand*{\QED}{\hfill\ensuremath{\square}}

%Estructura de ecuaciones
\setlength{\textwidth}{15cm} \setlength{\oddsidemargin}{5mm}
\setlength{\textheight}{23cm} \setlength{\topmargin}{-1cm}



\author{David García Curbelo}
\title{Probabilidad}
\date{Problema P3 Tema 4 Propuesto}

\pagestyle{empty}



\def\R{\mathds{R}}
\def\sup{$^2$}

\begin{document}
    \maketitle
    \setcounter{page}{1}
    \pagestyle{plain}
    
    \textbf{Ejercicio: } \textit{Demostrar las propiedades enunciadas en el P3T4 }\\ \\

    Calculemos para ello los siguientes datos, necesarios para la demostración del primer apartado:
$$
    E[X/Y=1]=3/2, \quad
    E[X/Y=2]=9/5, \quad 
    E[X/Y=3]=20/7, \quad 
    E[X/Y=4]=16/5 
$$

$$
    E[X/Y]= \left\{
    \begin{aligned}
        1 \xrightarrow{f_2} 3/2  ,\quad P_{f_2}(1)=2/24 \quad\\
        2 \xrightarrow{f_2} 9/5 ,\quad P_{f_2}(2)=5/24 \quad\\
        3 \xrightarrow{f_2} 20/7 ,\quad P_{f_2}(3)=7/24 \thinspace \thinspace\\
        4 \xrightarrow{f_2} 16/5 ,\quad P_{f_2}(4)=10/24\\
    \end{aligned}
    \right.
$$

$$
    E[X^2/Y=1]=5/2, \quad
    E[X^2/Y=2]=19/5, \quad 
    E[X^2/Y=3]=62/7, \quad 
    E[X^2/Y=4]=106/10 
$$

$$
    E[X^2/Y]= \left\{
    \begin{aligned}
        1 \xrightarrow{h_2} 5/2  ,\quad P_{h_2}(1)=2/24 \quad\\
        2 \xrightarrow{h_2} 19/5 ,\quad P_{h_2}(2)=5/24 \quad\\
        3 \xrightarrow{h_2} 62/7 ,\quad P_{h_2}(3)=7/24 \thinspace \thinspace\\
        4 \xrightarrow{h_2} 106/10 ,\quad P_{h_2}(4)=10/24\\
    \end{aligned}
    \right.
$$


    
    \begin{align*}
        &Var[X/Y=1] = 5/2 - (3/2)^2 = 1/4, \quad  P_Y(1) = 2/24\\
        &Var[X/Y=2] = 19/5 - (9/5)^2 = 14/25, \quad  P_Y(2) = 5/24\\
        &Var[X/Y=3] = 62/7 - (20/7)^2 = 34/49, \quad  P_Y(3) = 7/24 \\
        &Var[X/Y=4] = 106/10 - (16/5)^2 = -176/25, \quad  P_Y(4) = 10/24\\
    \end{align*}

    $$ECM[Var(X/Y)] = \frac{1}{4} \cdot \frac{2}{24} + \frac{14}{25} \cdot \frac{5}{24} + \frac{34}{49} \cdot \frac{7}{24} + \frac{9}{25} \cdot \frac{10}{24} = \frac{823}{1680}$$
 
    \newpage

    Resolvamos por tanto el siguiente apartado:

    \begin{align*}
        &E[X] = \frac{1}{8} + 2\cdot\frac{7}{24} + 3\cdot\frac{3}{8} + 4 \cdot \frac{5}{24} = \frac{8}{3} \\
        &E[X^2] = \frac{1}{8} + 2^2\cdot\frac{7}{24} + 3^2\cdot\frac{3}{8} + 4^2\cdot \frac{5}{24} = 8 
    \end{align*}

    Y por lo tanto obtenemos $Var[X] = E[X^2] - \left(E[X]\right)^2 = \frac{8}{9}$

    \begin{align*}
        &E[Y] = \frac{1}{12} + 2\cdot\frac{5}{24} + 3\cdot\frac{7}{24} + 4 \cdot \frac{5}{12} = \frac{73}{24} \\
        &E[Y^2] = \frac{1}{12} + 2^2\cdot\frac{5}{24} + 3^2\cdot\frac{7}{24} + 4^2\cdot \frac{5}{12} = \frac{245}{24}
    \end{align*}

    Y por lo tanto obtenemos $Var[Y] = E[Y^2] - \left(E[Y]\right)^2 =\frac{551}{576}$. Entonces tenemos que:

    \begin{align*}
        &\eta ^2 _{Y/X} = 0.485887
        &\eta ^2 _{X/Y} = 0.448884
    \end{align*}

    \newpage

    Calculemos las rectas de regresión para el siguiente apartado:
    $$E[XY] = \frac{209}{24}, \quad Cov[X,Y] = E[XY] - E[X]\cdot E[Y] = \frac{43}{72}$$

    Entonces obtenemos:\\

    \fbox{X/Y}\\
    $$\hat{x} = E[X] + \frac{Cov(X,Y)}{Var[Y]}(Y-E[Y]) = \frac{344}{551}x + \frac{423}{451}$$\\

    \fbox{Y/X}\\
    $$\hat{y} = E[Y] + \frac{Cov(X,Y)}{Var[X]}(X-E[X]) = \frac{43}{64}x + \frac{5}{4}$$\\ \\

    Por último calcularemos el coeficiente de correlación
    $$P_{X,Y} = \frac{Cov(X,Y)}{\sqrt{Var[X]Var[Y]}} = \frac{43/72}{\sqrt{\frac{551}{576}\cdot \frac{8}{9}}} = 0.164766$$
    El cual podemos apreciar que se trata de un buen ajuste, ya que el coeficiente de correlación es próximo a 1.







\end{document}