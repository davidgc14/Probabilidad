\documentclass[a4paper, 12pt]{article}

\usepackage{mathexam}
\usepackage{amsmath}
\usepackage{amsfonts}
\usepackage{graphicx}
\usepackage{systeme}
\usepackage{microtype}
\usepackage{multirow}
\usepackage{pgfplots}
\usepackage{listings}
\usepackage{tikz}
\usepackage{dsfont} %Numeros reales, naturales...
\usepackage{verbatimbox} %comentarios

%\graphicspath{{images/}}
\newcommand*{\qed}{\hfill\ensuremath{\square}}

%Estructura de ecuaciones
\setlength{\textwidth}{15cm} \setlength{\oddsidemargin}{5mm}
\setlength{\textheight}{23cm} \setlength{\topmargin}{-1cm}



\author{David García Curbelo}
\title{Probabilidad}
\date{Problema P16 Propuesto}

\pagestyle{empty}


\def\R{\mathds{R}}
\def\sup{$^2$}

\begin{document}
    \maketitle
    \setcounter{page}{1}
    \pagestyle{plain}

    {\textit{El vector aleatorio $(X, Y)$ se distribuye según una uniforme sobre el recinto:
    $$R=\{(x,y); 0<x<y<1\}$$
    Calcular:
    \begin{itemize}
        \item Su función generatriz de momentos conjunta.
        \item Las distribuciones generatrices de momentos marginales.
        \item La covarianza de X e Y.\\ \\
    \end{itemize}
    }}

    {\textbf{Función generatriz de momentos conjunta}}
    Calculemos primero su función de densidad, la cual viene dada por 
    $$f_{(X,Y)}(x,y)=2 \cdot 1_{[R]}(x,y)=2; \quad 0<x<y<1$$
    Sabemos que la función generatriz de momentos conjuntos se calcula mediante
    $$M_{X,Y}(t_1,t_2)=E\left[exp(t_1 X + t_2Y)\right]$$
    Como se trata de vector aleatorio continuo, el calculo de la esperanza se calcula
    de la siguiente forma
    $$E\left[exp(t_1 X + t_2Y)\right]=2\int _0^1 \int_x^1 exp(t_1x+t_2y)dydx = 2\int _0^1 e^{t_1x}\int_x^1 e^{t_2y}dydx = 
    $$$$\frac{2}{t_2}\int _0^1 e^{t_1x} \left[e^{t_2}-e^{t_2x}\right] dx = \frac{2}{t_2}\int _0^1 e^{t_1x+t_2} -e^{x(t_1+t_2)} dx$$
    $$\Rightarrow M_{X,Y}(t_1,t_2)=2\frac{(t_1+t_2)e^{t_2}[e^{t_1}-1]+t_1[1-e^{t_1+t_2}]}{t_1t_2(t_1+t_2)}, \quad \forall (t_1,t_2) \in \R^2$$
    
    Usando el desarrollo de Taylor visto en clase, llegamos a la expresión
    $$M_{(X,Y)}(t_1,t_2) = 2 \left[\sum_{k=1}^{\infty} \frac{t_1^{k-1}}{k!} + \sum_{l=1}^{\infty} \frac{t_1^{l-1}t_2^k}{(k+1)!l!} - \sum _{u=1}^{k-1} \frac{\binom{k-1}{u} t_1^{k-1-u}t_2^{u-1}}{k!}\right]$$
    
    \newpage

    {\textbf{Distribuciones generatrices de momentos marginales}}

    Calculamos primero la marginal de X. Dicha marginal viene dada por 
    $$M_X(t_1)=M_{X,Y} (t_1,0), \forall t_1 \in \R.$$
    
    Notemos que la sustitución $t_2=0$ en $M_{X,Y}(t_1,t_2)$ puede producir una indeterminación, por ello tengamos 
    en cuenta que la función generatriz de momentos es derivable y, por tanto, continua. Así, obtenemos
    $$M_X(t_1)=\lim_{t_2\rightarrow 0}M_{X,Y} (t_1,t_2)= \lim_{t_2\rightarrow 0} 2 \left[\sum_{k=1}^{\infty} \frac{t_1^{k-1}}{k!} + \sum_{l=1}^{\infty} \frac{t_1^{l-1}t_2^k}{(k+1)!l!} - \sum _{u=1}^{k-1} \frac{\binom{k-1}{u} t_1^{k-1-u}t_2^{u-1}}{k!}\right] = $$
    $$= 2\sum_{k=1}^{\infty} \frac{t_1^{k-1}}{k!} =\frac{2}{t_1} \left[ \sum_{k=0}^{\infty} \frac{t_1^{k}}{k!} -1\right] = \frac{2}{t_1}[e^{t_1}-1]
     \quad \forall t_1 \in \R$$

    Procedemos al cálculo de la marginal de $Y$ mediante el mismo proceso. Dicha marginal viene dada por
    $$M_Y(t_2)=M_{X,Y} (0,t_2), \forall t_2 \in \R.$$
    
    Observamos que en todos los términos de la función $M_{X,Y} (t_1,t_2)$ aparece el término $t_1$, por tanto es fácil ver que
    $$M_Y(t_2)=\lim_{t_1\rightarrow 0}M_{X,Y} (t_1,t_2)= $$$$\lim_{t_1\rightarrow 0} 2 \left[\sum_{k=1}^{\infty} \frac{t_1^{k-1}}{k!} + \sum_{l=1}^{\infty} \frac{t_1^{l-1}t_2^k}{(k+1)!l!} - \sum _{u=1}^{k-1} \frac{\binom{k-1}{u} t_1^{k-1-u}t_2^{u-1}}{k!}\right]
    = 0,\quad \forall t_2 \in \R$$

    \newpage

    {\textbf{Covarianza de X e Y}}
    
    La covarianza de dos variables aleatorias viene dado por 
    $$Cov(X,Y)=E[XY]-E[X]E[Y]$$
    Recordamos que la esperanza se define como
    $$E[g(x,y)]=\int _{-\infty}^{\infty} \int _{-\infty}^{\infty} g(x,y) f_{(X,Y)}(x,y)dydx$$\\ \\

    $$\Rightarrow E[XY]= \int _{-\infty}^{\infty} \int _{-\infty}^{\infty} 2xy dydx = \int _{0}^{1} \int _{x}^{1} 2xy dydx = \int _{0}^{1} x-x^3dx=\frac{1}{4}$$
    $$E[X]=\int _{0}^{1} \int _{x}^{1} 2x dydx = \int _{0}^{1} 2x-2x^2 dx = \frac{1}{3}$$
    $$E[Y]=\int _{0}^{1} \int _{x}^{1} 2y dydx = \int _{0}^{1} 1-x^2 dx = \frac{2}{3}$$\\ \\

    $$\Rightarrow Cov(X,Y)=\frac{1}{4}-\frac{2}{9}=\frac{1}{36}$$
\end{document}