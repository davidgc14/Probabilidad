\documentclass[fleqn]{article}

%\pgfplotsset{compat=1.17}

\usepackage{mathexam}
\usepackage{amsmath}
\usepackage{amsfonts}
\usepackage{graphicx}
\usepackage{systeme}
\usepackage{microtype}
\usepackage{multirow}
\usepackage{pgfplots}
\usepackage{listings}
\usepackage{tikz}
\usepackage{dsfont} %Numeros reales, naturales...

%\graphicspath{{images/}}
\newcommand*{\QED}{\hfill\ensuremath{\square}}

%Estructura de ecuaciones
\setlength{\textwidth}{15cm} \setlength{\oddsidemargin}{5mm}
\setlength{\textheight}{23cm} \setlength{\topmargin}{-1cm}



\author{David García Curbelo}
\title{Probabilidad}
\date{Apartados propuestos de P10 y P11}

\pagestyle{empty}



\def\R{\mathds{R}}
\def\sup{$^2$}

\begin{document}
    \maketitle
    \setcounter{page}{1}
    \pagestyle{plain}

    \textbf{Ejercicio: } \textit{Calcular y sustituir en las fórmulas anteriores el valor del coeficiente de correlación lineal $\rho_{X,Y}$} \\ 

    Calculemos primero $\rho^2_{X,Y}$, la cual sabemos calcular de la siguiente forma:

    $$\rho_{X,Y}^2 = \frac{ \left[ \int_0^{\infty} \int_x^{\infty} \right]^2 }{}$$

\end{document}